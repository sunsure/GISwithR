\documentclass{beamer}

% ASDA with R by Roger Bivand
% KML type
% RworldMaps by Any South
% ggmap - hadley Wickham

%-----------------------------------%
\usepackage{default}



\begin{document}
%-----------------------------------%
\begin{frame}[fragile]
\begin{verbatim}
library(ggmap)
\end{verbatim}
\end{framed}
%-----------------------------------%
% ggmpa
%-----------------------------------%
\begin{frame}[fragile]
\begin{verbatim}
library(ggmap)
\end{verbatim}
\end{framed}
%-----------------------------------%
%-----------------------------------%
\begin{frame}[fragile]
\begin{verbatim}
library(RgoogleMap)
\end{verbatim}
\end{framed}

install.packages("mapdata")
ins

library(maps)
library(mapdata)
map("worldHires","Mexico"
 xlim=c(-118.4,-86.7)
 ylim=c(14.5321,32.71865),
 col="blue",fill=TRUE)


library(maptools) #for shapefiles
‘readShapePoly’
Read in a polygon shape layer (e.g., administrative
boundaries, national parks, etc.). This means the layer is of
type “polygon” (i.e. not lines, such as roads or rivers)
‘readShapeLines’
read in a line shape layer
‘readShapePoints’
read in a point shape layerlibrary(maptools) #for shapefiles
‘readShapePoly’


%-----------------------------------%
\end{document}
