%-----------------------------------------------%
3.12 Vector Fields
3.17 Creating a Map
3.18 Chloropleth maps
3.20 Shapefiles

%-----------------------------------------------%
%Page 309
The default projection is the Mercator projection

\begin{verbatim}
library(maps)
states_map <- map_data("state")
ggplot(states_map, aes(x=long,y=lat,group=group)) + geom_polygon(fill="white",colour="black")
\end{verbatim}

%-----------------------------------------------%


The \texttt{map-uscore-data()} function

This is a grouping variable for each polygon. A region or subregions might have multiple polygons (i.e. if islands are
included)

%----------------------------------------------%

% Page 311
It is possibel to get speciific regions from a particular map.

\begin{verbatim}
east_asia <- map_data("world",region=c("japan","China","North Korea","South Korea"))

#Map regions to fill colour

scale_fill_brewer("pallette_set2")
\end{verbatim}
%----------------------------------------------%
\begin{verbatim}
NZ1 <- map_data("world")

NZ2 <- map_data("nz")
ggplot(nz2 (aes(x=long,y=lat)) + geom_path().
\end{verbatim}
%----------------------------------------------%
%Section 3.18
%PAge 313
Creating a Chlorpleth Map


