%-----------------------------------------------%
13.12 Vector Fields
13.17 Creating a Map
13.18 Chloropleth maps
13.19 
13.20 Shapefiles
Wheatley

%-----------------------------------------------%
www.gadm.org

%-----------------------------------------------%
%Page 294
Creating a vector field
use \texttt{geom_segment}. Lets use the \texttt{isabel} data set.


library{gcookbook)
islabel

x and y are longtitude and latitude values respectively.
Z is the height in kilometres.

The vx,vy and vz are windspeed components in each of the directions (m/s).
Speed is the windspeed.

\begin{framed}
\begin{verbatim}


\end{verbatim}
\end{framed}

%Page 297
We can map speed to colour. We'll also ass a map of 
the USA and zoom in on the area of interest,as shown in the graph on the right.

%-----------%
%PAge 298
\begin{verbatim}
usa <- map_data("usa")
scale_colour_continuous(low="grey80",high="darkred")+
geom_path(aes(x=long,y=lat,group=group),data=usa)+
ggplot(islicesub, aes(x=x,y=y)) + geom_segment(aes(xend =x +vx/50,vend=y+vy/50,colour=speed),
\end{verbatim}
%-----------%
%PAge 299
The \textbf{\textit{isabel}} data set has three-dimensional, so we can make a faceted graph of the data.

\begin{verbatim}
#Page 299
ggplot(isub,aes(x=x, y=y)) + + geom_segment(aes(xend =x +vx/50,vend=y+vy/50,colour=speed),
scale_colour_continuous(low="grey80",high="darkred")+
facet_wrap(-z)
\end{verbatim}
%-----------------------------------------------%
%Section 13.17
%-----------------------------------------------%
%-----------------------------------------------%
%Page 309 
Creating a map.

The default projection is the Mercator projection

\begin{verbatim}
library(maps)
states_map <- map_data("state")
ggplot(states_map, aes(x=long,y=lat,group=group)) + geom_polygon(fill="white",colour="black")
\end{verbatim}

%-------------%
%Page 309 - bottom section
The \texttt{map-uscore-data()} function

\begin{itemize}
\item[long] Longtitude
\item[lat] Latitude
\item[group] This is a grouping variable for each polygon. A region or subregion might have multiple polygons.
%---------%
%Page 310 - two maps + two paragraphs
\item[order] The order to connect each point in a group
\item[region] Roughly, the names of countries, although some other objects are lakes.
%---------%
%Page 311
\item[subregion] The names of subregions with a region, which can be obtained for multiple groups
(For example, the Alaska subregion includes many islands, each within its own group.
\end{itemize}
%---------%
There are a number of different maps available, including world, nz, france,italy,usa (outline of USA),state (each state in the USA) and county (each county in the USA).


To get a map of the world
\begin{framed}
\begin{verbatim}
world_map <- map_data("world")
world_map
\end{verbatim}
\end{framed}

%---------%
%page 311 - mid section
If you want to draw a map of a region in the world map for which there isn't a separate map, you can
first look for the region name

\begin{framed}
\begin{verbatim}
sort(unique(world_map$region))

east_asia <- map_data("world",region=c("Japan","China","North Korea","South Korea")).

\end{verbatim}
\end{framed}
%----------------------------------------------------%
# Page 311

\begin{framed}
\begin{verbatim}
ggplot(east_asia, aes(x=long,y=lat,group=group,fill=region)) + 
geom_polygon(colours="black")+
scale_fill_brewer(pallette="set2)
\end{verbatim}
\end{framed}


%----------------------------------------------%

% Page 311 - Bottom of page
It is possibel to get speciific regions from a particular map.

\begin{framed]
\begin{verbatim}
east_asia <- map_data("world",region=c("japan","China","North Korea","South Korea"))

#Map regions to fill colour

scale_fill_brewer("pallette_set2")
\end{verbatim}
\end{framed}
%----------------------------------------------%

%Page 312 (TOP OF PAGE)

If there is a separae map for each region, that map data will be at
higher resolution that in youw ere to extrac it from the ``world" map.


%---------------%
\begin{framed}
\begin{verbatim}
NZ1 <- map_data("world",region="New Zealand")
NZ1 < subset(NZ1, long>0 &lat> -48)  #trim off outlying islands
ggplot(NZ1,aes(x=long,y=lat,group=group)) + geom_path()

NZ2 <- map_data("nz")
ggplot(nz2 (aes(x=long,y=lat)) + geom_path().
\end{verbatim}
\end{framed}
%---------------%

%Page 312 - Discussion



%----------------------------------------------%
%Page 313 - TWO NZ graps

%Section 13.18 / Creating a Chlorpleth Map

What is a Chloropleth map?

%PAge 313 Mid Page

The Crime Data (USArrests)

%---------------%
\begin{framed}
\begin{verbatim}
crimes <-data.frame(state =tolower(rownames(USArrests)),USArrests)
head(crimes)
\end{verbatim}
\end{framed}
%---------------%

%PAge 313/314

%Page 314 is good shape
Creating a map where regions are colours according to some variable value.
\begin{verbatim}
library(maps)
states_map <- map_data("state")
crime_map <- merge(states_map,crimes,by.x="region",by.y="state")
head(crime_map)

\end{verbatim}

After mergeing, the order has changed, leading to polygons drawn in the wrong order.
To solve this, the data must be sorted.
%-----------------------------------------------------%


%Page 314 Plyr
\begin{framed}
\begin{verbatim}
libary(plyr)
# Use this package for the arrange function().
crime_map <- arrange(crime_map,group,order)
head(crime_map)
\end{verbatim}
\end{framed}

%R output next

%------------%
% Page 314 (Mid page)

\begin{verbatim}
ggplot(crimes, aes(map_id=state,fill=Assault))+ 
geom_polygon(colour="black")+
expand_limits(x=states_map$long,y=states_map$lat) + 
coord_map("polyconic")

\end{verbatim}


# Chloropleth Maps
\begin{verbatim}
##############
# PAge 314
ggplot(crimes(aes(map_id=state,fill=Assault))+
geom_map(map=state_map,colour="black") + 
scale_fill_gradient2(low="#559999",mid="grey90",high="#BB650B",midpoint=median(crimes$Assault)) +

expand_limits(x=state_map$long,y=state_map$lat) +
co_ordmap("poyconic")

############
\end{verbatim}

%End of PAge 314. Two graphs on next page
%------------------------%
% Page 315 - very Little Content on this page
% 2 Diagrams

The previous example mapped continuous variables to \texttt{fill}, but discrete values could have also have been used. 
It is often easier to interpret the data if the data has been discretized.
For example, we could categorize the values into quantiles and show thos quantiles.
(Next Example)

%End of Page 315




%------------------------%

##############
# Page 316/7

ggplot(crimes(aes(map_id=state,fill=Assault))+
geom_map(map=state_map,colour="black") + 
expand_limits(x=state_map$long,y=state_map$lat) +
co_ordmap("poyconic")

############
%------------------------------------------------%

%Page317
Making a map with a clean background : removing background elements from a map.

%-----------------------%
% Page 317/8
#Create a theme with many of the background elements removed.
theme_clean <-function(base_size=12){
require(grid)
there_grey(base_size) %+replace%
theme(
axis.title =element_blank(),
axis.text =element_blank(),
panel.background =element_blank(),
panel.grid =element_blank(),
axis.tick.lengths=unit(0,"cm"),
axix.tick.margins=unit(0,"cm"),
panel.margins=unit(0,"cm"),
plot.margin=unit(0,"cm"),
complete=TRUE
)
}

\begin{framed}
\begin{verbatim}
##############
ggplot(crimes(aes(map_id=state,fill=assault_q))+
geom_map(map=state_map,colour="black") + 
scale_fill_manual(values=pal)+
expand_limits(x=state_map$long,y=state_map$lat) +
co_ordmap("poyconic") + 
labs(fill="Assault Rate \n Percentile")+
theme_clean()

############
\end{verbatim}
\end{framed}

%------------------------------------------------%
%PAge 319 Top of page

In some maps it is important to include cotnextual infromation such as longitutda end latitude.
 
%Page 319
%------------------------------------------------%
%PAge 319
%13.20 Creating a map from a shapefile

Creating a map from an ESRI shapefile.

%------------------------------------------------%
Page 319
Creating a map from an ESRI Shapefile

load the shapefile using readShapePoly() function from the map tools package.
Convert it to a dataframe using the command fortify
then plot it.

\begin{verbatim}

taiwan_shp <-readShapePoly(".....shp")
taiwan_map <-fortify(taiwan_shp)

ggplot(taiwan_map, aes(x=long,y=lat,group=group)) + geom_path()

\end{verbatim}

%------------------------------------------------%
%PAge 319
%13.20 Creating a map from a shapefile

Creating a map from an ESRI shapefile.


%------------------------------------------------%
%Page 321
It is actually possible to pass athe SpatialPolygonDataFrame objects directly to ggplot() command, which will automatically ``fortify" it.

Doing the operation into stages allows the user to inspect the data first, and also to merge it with other data.

%------------------------------------------------%
%Wheatley
Packages
\begin{itemize}
\item sp vector data
\item raster grid data
\item rgdal input/output, coordinate systems
\item rgeos geometric calculations on vector data
\end{itemize}
%-------------------------------------------------%
Vector example: admin boundary
\begin{verbatim}
map <- getData('GADM',country='IRL', level=0)
projection(map)
# [1] "+proj=longlat +datum=WGS84"
\end{verbatim}
%------------------------------------------------%

Vector example: soil map of world
ESRI shapefile from fao.org
soils <- readOGR("~/DSMW/", "DSMW") #read
projection(soils) <- "+proj=longlat +datum=WGS84"
dim(soils@data)
#[1] 34112 12
soils@data[5000:5002,]
%------------------------------------------------%


Black Earth Soils
\begin{verbatim}
blackearth <- subset(soils, substr(DOMSOI,1,1) %in% c("C","K"))
>blackearth <- gUnaryUnion(blackearth)
\end{verbatim}
