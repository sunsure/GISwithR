%-----------------------------------------------%
13.12 Vector Fields
13.17 Creating a Map
13.18 Chloropleth maps
13.19 
13.20 Shapefiles
%-----------------------------------------------%
%Page 294
Creating a vector field
use \texttt{geom_segment}. Lets use the \texttt{isabel} data set.


library{gcookbook)
islabel

x and y are longtitude and latitude values respectively.
Z is the height in kilometres.

The vx,vy and vz are windspeed components in each of the directions (m/s).
Speed is the windspeed.

\begin{framed}
\begin{verbatim}


\end{verbatim}
\end{framed}

%Page 297
We can map speed to colour. We'll also ass a map of 
the USA and zoom in on the area of interest,as shown in the graph on the right.

%-----------%
%PAge 298
\begin{verbatim}
usa <- map_data("usa")
scale_colour_continuous(low="grey80",high="darkred")+
geom_path(aes(x=long,y=lat,group=group),data=usa)+
ggplot(islicesub, aes(x=x,y=y)) + geom_segment(aes(xend =x +vx/50,vend=y+vy/50,colour=speed),
\end{verbatim}
%-----------%
%PAge 299
The \textbf{\textit{isabel}} data set has three-dimensional, so we can make a faceted graph of the data.

\begin{verbatim}
#Page 299
ggplot(isub,aes(x=x, y=y)) + + geom_segment(aes(xend =x +vx/50,vend=y+vy/50,colour=speed),
scale_colour_continuous(low="grey80",high="darkred")+
facet_wrap(-z)
\end{verbatim}
%-----------------------------------------------%
%-----------------------------------------------%
%Page 309


The default projection is the Mercator projection

\begin{verbatim}
library(maps)
states_map <- map_data("state")
ggplot(states_map, aes(x=long,y=lat,group=group)) + geom_polygon(fill="white",colour="black")
\end{verbatim}

%-------------%
The \texttt{map-uscore-data()} function

\begin{itemize}
\item[long] Longtitude
\item[lat] Latitude
\item[group] This is a grouping variable for each polygon. A region or subregion might have multiple polygons.
\item[order] The order to connect each point in a group
\item[region] Roughly, the names of countries, although some other objects are lakes.
\end{itemize}
%-----------------------------------------------%




%----------------------------------------------%

% Page 311
It is possibel to get speciific regions from a particular map.

\begin{verbatim}
east_asia <- map_data("world",region=c("japan","China","North Korea","South Korea"))

#Map regions to fill colour

scale_fill_brewer("pallette_set2")
\end{verbatim}
%----------------------------------------------%
\begin{verbatim}
NZ1 <- map_data("world")

NZ2 <- map_data("nz")
ggplot(nz2 (aes(x=long,y=lat)) + geom_path().
\end{verbatim}
%----------------------------------------------%
%Section 3.18
%PAge 313
Creating a Chlorpleth Map


