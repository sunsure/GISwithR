\section{Using Raster Data} 
\begin{verbatim}
 
### USING RASTER (GRID) DATA ####
 
 
### DOWNLOADING DATA
tmin <- getData("worldclim", var="tmin", res=10)   # this will download global
data on minimum temperature at 10 min resolution
  # can also get other climatic data, elevation, administrative boundaries, etc
 
### LOADING A RASTER LAYER
tmin1 <- raster("~/UsingR-GIS/wc10/tmin1.bil")   # Tmin for January
fromDisk(tmin1)  # values are stored on disk instead of memory! (useful for
large rasters)
tmin1 <- tmin1/10    # Worldclim temperature data come in decimal degrees
tmin1    # look at the info
plot(tmin1)
 
?raster    # raster reads many different formats, including Arc ASCII grids or
netcdf files
 
 
 
### CREATING A RASTER STACK (collection of many raster layers with the same
projection, spatial extent and resolution)
library(gtools)
list.ras <- mixedsort(list.files("~/UsingR-GIS/wc10/", full.names=T,
pattern=".bil"))
list.ras   # I have just collected a list of the files containing monthly
temperature values
tmin.all <- stack(list.ras)
tmin.all
tmin.all <- tmin.all/10
plot(tmin.all)
 
# BRICKS
tmin.brick <- brick(tmin.all)   # a rasterbrick is similar to a raster stack
(i.e. multiple layers with the same extent and resolution), but all the data
must be stored in a single file
 
 
 
### CROP RASTERS
plot(tmin1)
newext <- drawExtent()    # click on the map
tmin1.c <- crop(tmin1, newext)
plot(tmin1.c)
 
newext2 <- c(-10, 10, 30, 50)   # alternatively, provide limits
tmin1.c2 <- crop(tmin1, newext2)
plot(tmin1.c2)
 
tmin.all.c <- crop(tmin.all, newext)
plot(tmin.all.c)
 
\end{verbatim}

\subsection{Projections}
\begin{verbatim} 
 
### DEFINE PROJECTION
crs.geo    # defined above
projection(tmin1.c) <- crs.geo
projection(tmin.all.c) <- crs.geo
tmin1.c    # notice info info at coord.ref.
 
### CHANGING PROJECTION
tmin1.proj <- projectRaster(tmin1.c, crs="+proj=merc +lon_0=0 +k=1 +x_0=0
+y_0=0 +a=6378137 +b=6378137 +units=m +no_defs")
tmin1.proj   # notice info info at coord.ref.
plot(tmin1.proj)
# can also use a template raster, see ?projectRaster
 
 
 
### PLOTTING
histogram(tmin1.c)
pairs(tmin.all.c)
persp(tmin1.c)
contour(tmin1.c)
contourplot(tmin1.c)
levelplot(tmin1.c)
plot3D(tmin1.c)
bwplot(tmin.all.c)
densityplot(tmin1.c)
 
 
 
### Spatial autocorrelation
Moran(tmin1.c)    # global Moran's I
tmin1.Moran <- MoranLocal(tmin1.c)
plot(tmin1.Moran)
 
 
### EXTRACT VALUES FROM RASTER
View(locs)    # we'll obtain tmin values for our points
locs$tmin1 <- extract(tmin1, locs)    # values are incorporated to the
dataframe
View(locs)
 
# extract values for a given region
plot(tmin1.c)
reg.clim <- extract(tmin1.c, drawExtent())
summary(reg.clim)
 
# rasterToPoints
tminvals <- rasterToPoints(tmin1.c)
View(tminvals)
 
## CLICK: get values from particular locations in the map
plot(tmin1.c)
click(tmin1.c, n=3)   # click n times in the map
 
 
 
### RASTERIZE POINTS, LINES OR POLYGONS
locs2ras <- rasterize(locs.gb, tmin1)
locs2ras
plot(locs2ras, xlim=c(-10,10), ylim=c(45, 60), legend=F)
plot(wrld_simpl, add=T)
 
 
### CHANGING RESOLUTION (aggregate)
tmin1.lowres <- aggregate(tmin1.c, fact=2, fun=mean)
tmin1.lowres
tmin1.c     # compare
par(mfcol=c(1,2))
plot(tmin1.c, main="original")
plot(tmin1.lowres, main="low resolution")
dev.off()
 
 
  \end{verbatim}
