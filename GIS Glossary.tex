For its reference system, KML uses 3D geographic coordinates: longitude, latitude and altitude, in that order, with negative values for west, south and below mean sea level if the altitude data is available. The longitude, latitude components are as defined by the World Geodetic System of 1984 (WGS84). The vertical component (altitude) is measured from the WGS84 EGM96 Geoid vertical datum. If altitude is omitted from a coordinate string, e.g. (−122.917, 49.2623) then the default value of 0 (approximately sea level) is assumed for the altitude component, i.e. (−122.917, 49.2623, 0). A formal definition of the coordinate reference system (encoded as GML) used by KML is contained in the OGC KML 2.2 Specification. This definition references well-known EPSG CRS components.
